\section{Introduction}



\section{Etudes du dataset}
En visualisant le dataset, nous avons tout d'abord commencé par étudié les différentes features. Certaines nous paraissaient indispensables, alors que certaines avaient l'air inutiles comme par exemples les intérêts personnels (sports, tvsports, ...) de la personne.
Parfois, il serait même plus utile de combiner certaines features entre elles afin d'en créer une plus réaliste comme par exemples l'intelligence que l'on porte à son partenaire et l'intelligence de celui-ci.
Vu la quantité de features que nous avons à disposition, un grand nombre d'entre elles seront ignorées afin de ne pas bruiter le résultat de la prédiction.

Après avoir analysé le contenu du dataset, nous nous sommes aperçus qu'il y avait plein de données manquantes ou null. Il nous faudra soit filtrer ces données ou alors boucher 


\section{Problèmes rencontrés}


\section{Conclusion}